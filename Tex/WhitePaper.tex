\documentclass[12pt]{article} % use larger type; default would be 10pt
\usepackage{graphicx,amsmath,subfigure} % support the \includegraphics command and options
\usepackage[pdftex]{color}
\usepackage{natbib}
\usepackage{authblk}


\addtolength{\oddsidemargin}{-.5in}%
\addtolength{\evensidemargin}{-.5in}%
\addtolength{\textwidth}{1in}%
\addtolength{\textheight}{1.3in}%
\addtolength{\topmargin}{-.8in}%

\begin{document}


\def\spacingset#1{\renewcommand{\baselinestretch}%
{#1}\small\normalsize} \spacingset{1}


%%%%%%%%%%%%%%%%%%%%%%%%%%%%%%%%%%%%%%%%%%%%%%%%%%%%%%%%%%%%%%%%%%%%%%%%%%%%%%


  \title{\bf Capital One White Page}
  \author[1]{Marcos Carzolio}
  \author[1]{Andy Hoegh}
  \author[1]{Xinran Hu}
  \author[1]{Lucas Roberts}
  \author[1]{Yuhyun Song}
\affil[1]{Department of Statistics, Virginia Tech}
 \maketitle

\bigskip
\begin{abstract}
\noindent
abstract here....
\end{abstract}

\noindent%
{\it Keywords:} 
\vfill

\newpage


\newcommand{\ac}[1]{[{\color{red}\ Andy Says: {\tt #1}}]}
\newcommand{\xc}[1]{[{\color{red}\ Xinran Says: {\tt #1}}]}
\newcommand{\lc}[1]{[{\color{red}\ Lucas Says: {\tt #1}}]}
\newcommand{\mc}[1]{[{\color{red}\ Marcos Says: {\tt #1}}]}
\newcommand{\yc}[1]{[{\color{red}\ Yuhyun Says: {\tt #1}}]}

\section{Overview of Solution} 
Our solution is based on...
\subsection{VP Questions}
\begin{enumerate}
\item What defines a relevant merchant for a customer?
\item How does this tool support the business team's objective for the program?
\item How would you educate your business partners on the value the tool provides?
\item How might you evolve your product over time?
\item What other strategies could your tool support for the credit card or bank business?
\end{enumerate}

\subsection{Model Considerations}
\begin{enumerate}
\item Spending patterns and preferences vary from customer to customer as well as by location, size of household, income level, ect...
\item Merchant attributes and merchandise attributes play a large role - some merchants are visited infrequently (discretionary / luxury purchases) while others are visited on a regular basis (coffee shops)
\item Past spend is a great predictor of future spend, but merchants already have this information and identifying previous spenders is not their priority
\item It is imperative to ensure that our business partners understand our solution, how it adds value and its limitations
\end{enumerate}

\subsection{Model Considerations}
\begin{enumerate}
\item Customer receives offer and redeems one $\rightarrow +\$5$
\item Customer receives offer and doesn't redeem $\rightarrow -\$3$
\item No offer extended $\rightarrow -\$1$
\end{enumerate}


\subsection{Judging}
\begin{enumerate}
\item Performance of the statistical model evaluated on a validation sample.
\item Creativity of the candidate variables and intuition behind selected model features
\item A written proposal, describing how your tool addressed the aforementioned ``VP questions"
\end {enumerate}
Teams will score up to 20 points in each category for a maximum score of 60 points, top 5 teams are semi-finalists.
\ac{Get to Work!}

\section{Data Preparation}
Much effort was taken to extract meaningful features from the data set.
\section{Model Building/Selection}
Several off the shelf classification tools were initially used, then they were refined to achieve better results in this case.
\section{Model Use Strategy}
Separate models are constructed for each merchant, then using a specified loss function a decision is made on whether coupons should be issued.  Conditional on a coupon being issued the set of coupons most likely to result in coupon use are issued \ac{There may be structure between the merchants such that the optimal strategy may not be to issue the coupons to the merchants with the highest marginal probabilities} 
\section{Conclusions/Other Potential Opportunities}
While the problem is framed through the lens of giving a customer a coupon, the actual modeling scenario is slightly different.  The assumed response of a coupon would be a behavior adjustment that may nudge a customer in a direction they wouldn't otherwise go.  
\end{document}





